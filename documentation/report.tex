\documentclass[10pt,twocolumn,letterpaper]{article}

\usepackage{cvpr}
% \usepackage{times}

\usepackage{tgbonum} % TeX Gyre Bonum
% if we need a smaller font, i.e. narrower then
% \usepackage{tgtermes} % TeX Gyre Termes
% for other fonts need LuaTeX (or at least XeTeX)

\usepackage{epsfig}
\usepackage{graphicx}
\usepackage{amsmath}
\usepackage{amssymb}


% Include other packages here, before hyperref.

% If you comment hyperref and then uncomment it, you should delete
% egpaper.aux before re-running latex.  (Or just hit 'q' on the first latex
% run, let it finish, and you should be clear).
\usepackage[pagebackref=true,breaklinks=true,letterpaper=true,colorlinks,bookmarks=false]{hyperref}

\cvprfinalcopy % *** Uncomment this line for the final submission

\def\cvprPaperID{****} % *** Enter the CVPR Paper ID here
\def\httilde{\mbox{\tt\raisebox{-.5ex}{\symbol{126}}}}

% Pages are numbered in submission mode, and unnumbered in camera-ready
\ifcvprfinal\pagestyle{empty}\fi
\begin{document}

%%%%%%%%% TITLE
\title{Super-Resolution with GANs}

\author{Nathanael Bosch\\
{\tt\small nathanael.bosch@tum.de}
\and
Thomas Grassinger\\
{\tt\small thomas.grassinger@tum.de}
\and
Jonas Kipfstuhl\\
{\tt\small jonas.kipfstuhl@tum.de}
\and
Pierre Springer\\
{\tt\small pierre.springer@tum.de}
%\and
%Team Member 5\\
%{\tt\small fifth@i1.org}
}


\maketitle
%\thispagestyle{empty}

\section{Introduction}
\label{sec:intro}

% The idea and problem statement

\section{Setup}
\label{sec:setup}

% what we used for our work

\subsection{Dataset}
\label{sec:data}

% our datasets

\subsection{Networks}
\label{sec:nets}

% the Networks, e.g. VGG16, VGG19, VGG16/19

\section{Results}
\label{sec:results}

% our results => main section
% discriminatro may be omitted
% better nets work better
% ...

\section{Conclusion}
\label{sec:conclusion}

% something about what we learned

\appendix
%%% Appendix
% include images
% include graphs

{
  \nocite{*}                      % also those references without \cite{·}
  \small
  \bibliographystyle{ieee}
  \bibliography{bib}
  
}



%%%%%%
% Text of the proposal
% this may be reused in parts.
%
% \section{Introduction} The goal of this project is to gain a deeper
% understanding of Generative Adversarial Networks. We will then apply
% them to the image super resolution problem.  %and try different
% applications, so we probably won't invent new applications or methods
% for GANs, but will try different architectures, loss functions and
% applications for GANs (see \cite{Overview}).  %Depending on the
% conflicts with the data acquisition, server problems and the time it
% takes to understand GANs, we want to use the structure of the network
% to work on different problems.  %As there are more than 250 papers
% about GANs in arXiv, we collected a small set of interesting topics
% (see \ref{rel_work}) which we will consider in our work.


%     \subsection{Related Works}
%     \label{rel_work}
%         \begin{itemize}
%         \item Generative Adversarial Networks \cite{Goodfellow2014}
%         \item Generative Adversarial Networks: An Overview
% \cite{Overview}
%         %\item Learning to Discover Cross-Domain Relations with
% Generative Adversarial Networks \cite{Kim2017}
%         %\item SegAN: Adversarial Network with Multi-scale L1 Loss for
% Medical Image Segmentation \cite{Kim2017}
%         \item Photo-Realistic Single Image Super-Resolution Using a
% Generative Adversarial Network \cite{SuperRes}
%         %\item Anti-Makeup: Learning A Bi-Level Adversarial Network
% for Makeup-Invariant Face Verification \cite{MakeUpGAN}
%         \end{itemize}

% As we did not get responses for all datasets, for now, we decided to
% work on the Photo-Realistic Single Image Super-Resolution using a
% Generative Adversarial Network \cite{SuperRes} paper. The main goal is
% to learn how to increase the resolution of images up to 4 times, while
% still creating \emph{good} images that look realistic. This is
% achieved by training Generative Adversarial Networks on downscaled
% versions of high resolution images. The \emph{generator} part of the
% GAN will do the actual task of super-resolution, while the
% \emph{discriminator} is trained to distinguish between real images and
% super-resolved images. We will compare the results of this approach to
% regular methods, such as bicubic extrapolation. The work will be
% mostly based on \cite{SuperRes}.

% \section{Dataset} Our initial work towards \emph{understanding GANs}
% will use the MNIST and CIFAR-10 datasets, which we already worked with
% in the lecture and exercises. We will not give detailed explanation to
% these datasets.  %We already contacted the staff of the Multimodal
% Brain Tumor Segmentation Challenge to get the brain tumor images but
% did not get a response yet.

% For the Image Super-Resolution \cite{SuperRes} we will work with the
% DIV2k dataset \cite{Agustsson_2017_CVPR_Workshops}.  The dataset
% consists of 1000 2K resolution images divided into: 800 images for
% training, 100 images for validation, 100 images for testing. As we do
% not have access to the 100 high resolution test images we will only
% use the 800+100 images provided, and split into train-validation-test
% accordingly.  The images are provided in both high and low resolution
% images for 2, 3, and 4 downscaling factors.  The structure of the
% given images is \cite{Agustsson_2017_CVPR_Workshops}:

% \begin{quote} For the high resolution images: 0001.png, 0002.png, ...,
% 1000.png\\ \\ For the downscaled images:\\ YYYYx2.png for downscaling
% factor x2; where YYYY is the image ID\\ YYYYx3.png for downscaling
% factor x3; where YYYY is the image ID\\ YYYYx4.png for downscaling
% factor x4; where YYYY is the image ID
% \end{quote} We also have access to the ImageNet data but will most
% likely use a pretrained network to make use of its features such as
% the VGG network \cite{VGG}, which is already available within
% torchvision. Detailed information about ImageNet can be found in
% \cite{imagenet_cvpr09}.

% \section{Methodology} Our initial work will setup on the cs241 course,
% which treated the basics of GANs and followed the work of
% \cite{Goodfellow2014}. In this stage, we want to understand what GANs
% are and how they work, as well as why people use them and which
% problems they were meant to solve. The cs241 course provides practical
% guidelines for initial implementations, but also allows us to modify
% the results and explore the many possibilities. We will start working
% with the MNIST and CIFAR-10 dataset by applying and comparing
% different GAN models such as fully connected GAN networks, convolution
% GAN networks, conditional CNN, and Adversarial Autoencoders to get a
% understanding of Generative Adversarial Networks.

% In a second step, after having understood the principles of GANs, we
% will reproduce the results of the super-resolution paper
% \cite{SuperRes} and try different modifications. These modifications
% include for example, changing the input features or choosing other
% pretrained models, such as different VGG layer models or even an
% entirely different model e.g.\,, AlexNet. Using our knowledge from the
% initial work on the MNIST and CIFAR data, we might also find other
% ways to improve our models performance compared to that of the
% paper. These tasks (working on layers, data engineering, work on
% pretrained models, further research) can be easily split up, which
% provides an efficient workflow.

% We would like to use the Google cloud service as presented in the
% lecture, but the local possibilities of the computer vision group
% might also be sufficient.  %If time permits we would like to dive
% deeper into other papers presented in \ref{rel_work}, and then
% implement and train such a more advanced GAN.

% \section{Outcome} Our target still remains: We want to
% \textit{understand} GANs!\\ Additionally we want to create a GAN which
% is able to transform low-resolution images into higher resolutions
% with similar or even better results as \cite{SuperRes}.


\end{document}