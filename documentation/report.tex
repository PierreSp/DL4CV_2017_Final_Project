\documentclass[10pt,twocolumn,letterpaper]{article}

\usepackage{cvpr}
\usepackage{tgbonum}
\usepackage{epsfig}
\usepackage{graphicx}
\usepackage{amsmath}
\usepackage{amssymb}

% Include other packages here, before hyperref.
\usepackage{graphicx}
\usepackage{subcaption}

% If you comment hyperref and then uncomment it, you should delete
% egpaper.aux before re-running latex.  (Or just hit 'q' on the first latex
% run, let it finish, and you should be clear).
\usepackage[pagebackref=true,breaklinks=true,letterpaper=true,colorlinks,bookmarks=false]{hyperref}

\cvprfinalcopy % *** Uncomment this line for the final submission

\def\cvprPaperID{****} % *** Enter the CVPR Paper ID here
\def\httilde{\mbox{\tt\raisebox{-.5ex}{\symbol{126}}}}

% Pages are numbered in submission mode, and unnumbered in camera-ready
\ifcvprfinal\pagestyle{empty}\fi

\graphicspath{{results/}}
\newlength{\imagewidth}


\begin{document}

%%%%%%%%% TITLE
\title{Super-Resolution with GANs}

\author{Nathanael Bosch\\
{\tt\small nathanael.bosch@tum.de}
\and
Thomas Grassinger\\
{\tt\small thomas.grassinger@tum.de}
\and
Jonas Kipfstuhl\\
{\tt\small jonas.kipfstuhl@tum.de}
\and
Pierre Springer\\
{\tt\small pierre.springer@tum.de}
%\and
%Team Member 5\\
%{\tt\small fifth@i1.org}
}


\maketitle
%\thispagestyle{empty}

\section{Introduction}
We call super-resolution (SR) the task of estimating a high-resolution
(HR) image from its low-resolution (LR) counterpart. Recent work with
optimization-based methods largely focuses on minimizing the mean
squared reconstruction error, which results in high peak
signal--to--noise ratios (PSNR) but very smooth pictures.

The authors of the paper\cite{LedigChristian2016PSIS} propose using a
generative adversarial network (GAN) with an improved loss function.
We reimplemented the paper and questioned the purpose of the
discriminator and the suggested loss function.

\section{GANs}
GANs consist of two different networks, a Generator Network and a
Discriminator Network. The concept behind this is that the generative
network estimates a super-resolved image from its LR version with the
goal to become highly similar to real images that the discriminator
network fails to distinguish.

Therefore we optimize the discriminator network $D_{\Theta_D}$ in an
alternating manner along with the generative network $G_{\Theta_G}$ to
solve the adversarial min-max problem:
\begin{align*}
  min_{\Theta_G} max_{\Theta_G} &\mathbb{E}_{I^{HR} \backsim p_{\text{train}}(I^{HR})} [\text{log} D_{\Theta_D}(I^{HR})] \\
  +&\mathbb{E}_{I^{HR} \backsim p_G(I^{LR})} [\text{log} (1-G_{\Theta_G}(I^{LR}))]
\end{align*}
The perceptual loss $l^{SR}$ we defined as weighted sum of a content loss and an discriminative loss component:
\begin{equation*}
  l^{SR}=\alpha l^{SR}_{MSE} + \beta l^{SR}_{VGG16_19/i.j} + \gamma l^{SR}_{D}
\end{equation*}
More precisely, the content loss components are defined as follows:
\begin{align*}
  l^{SR}_{M SE}       &= \frac{1}{r^2WH}\sum_{x=1}^{rW}\sum_{y=1}^{rH}(I^{HR}_{x,y}-G_{\theta_G}(I^{LR})_{x,y})^2 \\
  l^{SR}_{VGG1619/i,j}&=\frac{1}{W_{i,j}H_{i,j}} \sum_{x=1}^{W_{i,j}}\sum_{y=1}^{H_{i,j}}(\Phi_{i,j}(I^{HR})_{x,y}-\Phi_{i,j}(G_{\Theta_G}(I^{LR}))_{x,y}^2
\end{align*}
with downsampling factor $r$ and $W,H$ defining the tensor size.
Finally the discriminative loss is defined as follows:
\begin{equation*}  
  l^{SR}_{D}=\sum_{n=1}^N -\text{log}D_{\Theta_D}(G_{\Theta_G}(I^{LR}))
\end{equation*}
\section{Setup}
\label{sec:setup}

% what we used for our work

% \subsection{Dataset}
% \label{sec:data}

For training we used the PASCAL VOC Dataset\cite{pascal-voc-2012} with
more than 10,000 images as well as the NITRE
Dataset\cite{Agustsson_2017_CVPR_Workshops} with 800 images.
% our datasets

% \subsection{Networks}
% \label{sec:nets}

We used pretrained VGG networks in different configurations. The best
results were obtained when we used a VGG1619 network. We also
considered a VGG16 network. Although faster at training it did not
yield results of equal quality.

% the Networks, e.g. VGG16, VGG19, VGG16/19

\section{Results}
\label{sec:results}

% our results => main section
% discriminatro may be omitted
% better nets work better
% ...

% images
We investigated on the proposed loss by the authors
of~\cite{LedigChristian2016PSIS}, by training multiple networks with
only parts of the loss. The reults can be seen in figure~\ref{fig:comp}.


\begin{figure*}[h]
  \centering
  \subcaptionbox{VGG1619 perceptual, adversarial, image loss}[0.2\linewidth]{%
    \includegraphics[width=0.15\linewidth, keepaspectratio]{vgg1619_p_a_i}
  }
  \subcaptionbox{VGG1619 perceptual, adversarial loss}[0.2\linewidth]{%
    \includegraphics[width=0.15\linewidth, keepaspectratio]{vgg1619_p_a}
  }
  \subcaptionbox{VGG1619 perceptual, image loss}[0.2\linewidth]{%
    \includegraphics[width=0.15\linewidth, keepaspectratio]{vgg1619_p_i}
  }
  \subcaptionbox{VGG1619 perceptual loss}[0.2\linewidth]{%
    \includegraphics[width=0.15\linewidth, keepaspectratio]{vgg1619_p}
  }
  \subcaptionbox{VGG1619 adversarial, image loss}[0.2\linewidth]{%
    \includegraphics[width=0.15\linewidth, keepaspectratio]{vgg1619_a_i}
  }
  \subcaptionbox{VGG1619 image loss}[0.2\linewidth]{%
    \includegraphics[width=0.15\linewidth, keepaspectratio]{vgg1619_i}
  }
  \subcaptionbox{VGG19 perception, adversarial, image loss}[0.2\linewidth]{%
    \includegraphics[width=0.15\linewidth, keepaspectratio]{vgg19_p_a_i}
  }
  \subcaptionbox{VGG16 perception, adversarial, image loss}[0.2\linewidth]{%
    \includegraphics[width=0.15\linewidth, keepaspectratio]{vgg16_p_a_i}
  }
  \caption{Comparison of several loss configurations}
  \label{fig:comp}
\end{figure*}

The curves of the PSNR and SSIM (structural similarities) values
during training may be seen in figure~\ref{fig:plots}.
\begin{figure*}[h]
  \centering
  \captionsetup{}
  \subcaptionbox{}[0.4\linewidth]{%
    \includegraphics[width=0.4\linewidth, keepaspectratio]{IMG_20180206_113139_679.jpg}
  }
  \subcaptionbox{}[0.4\linewidth]{%
    \includegraphics[width=0.4\linewidth, keepaspectratio]{IMG_20180206_113143_300.jpg}
  }
  \subcaptionbox{}[0.4\linewidth]{%
    \includegraphics[width=0.4\linewidth, keepaspectratio]{IMG_20180206_113147_469.jpg}
  }
  \subcaptionbox{}[0.4\linewidth]{%
    \includegraphics[width=0.4\linewidth, keepaspectratio]{IMG_20180206_113150_955.jpg}
  }
  \subcaptionbox{}[0.4\linewidth]{%
    \includegraphics[width=0.4\linewidth, keepaspectratio]{IMG_20180206_113156_419.jpg}
  }
  \subcaptionbox{}[0.4\linewidth]{%
    \includegraphics[width=0.4\linewidth, keepaspectratio]{IMG_20180206_113200_011.jpg}
  }
  \subcaptionbox{}[0.4\linewidth]{%
    \includegraphics[width=0.4\linewidth, keepaspectratio]{IMG_20180206_113202_817.jpg}
  }
  \subcaptionbox{}[0.4\linewidth]{%
    \includegraphics[width=0.4\linewidth, keepaspectratio]{IMG_20180206_113206_502.jpg}
  }  
  \caption{Curves of metrics during training}
  \label{fig:plots}
\end{figure*}


We observed that those widely used metrics are not sufficient for our
purposes.  While they did not change anymore during training,
the visual appearence still improved.  This observation was also made
the authors of~\cite{LedigChristian2016PSIS}.

\section{Conclusion}
\label{sec:conclusion}
In our work we achieved very good results using the proposed loss and
the residual network. During our analysis we found the perceptual loss
to be crucial for achieving high performance, but when used on its own
it led to weird artifacts in the output images.

The removal of the discriminator had no impact on the resulting image
quality in our work, as the generator performs very well with the
other parts of the loss. This leads to huge difficultuies in training
the GAN.


% something about what we learned

\appendix
%%% Appendix
% include images
% include graphs

{
  \nocite{*}                      % also those references without \cite{·}
  \small
  \bibliographystyle{ieee}
  \bibliography{bib}
  
}
\end{document}