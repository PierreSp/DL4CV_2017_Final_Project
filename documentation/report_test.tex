\documentclass[10pt,twocolumn,letterpaper]{article}

\usepackage{cvpr}
\usepackage{tgbonum}
\usepackage{epsfig}
\usepackage{graphicx}
\usepackage{amsmath}
\usepackage{amssymb}

% Include other packages here, before hyperref.

% If you comment hyperref and then uncomment it, you should delete
% egpaper.aux before re-running latex.  (Or just hit 'q' on the first latex
% run, let it finish, and you should be clear).
\usepackage[pagebackref=true,breaklinks=true,letterpaper=true,colorlinks,bookmarks=false]{hyperref}

\cvprfinalcopy % *** Uncomment this line for the final submission

\def\cvprPaperID{****} % *** Enter the CVPR Paper ID here
\def\httilde{\mbox{\tt\raisebox{-.5ex}{\symbol{126}}}}

% Pages are numbered in submission mode, and unnumbered in camera-ready
\ifcvprfinal\pagestyle{empty}\fi
\begin{document}

%%%%%%%%% TITLE
\title{Super-Resolution with GANs}

\author{Nathanael Bosch\\
{\tt\small nathanael.bosch@tum.de}
\and
Thomas Grassinger\\
{\tt\small thomas.grassinger@tum.de}
\and
Jonas Kipfstuhl\\
{\tt\small jonas.kipfstuhl@tum.de}
\and
Pierre Springer\\
{\tt\small pierre.springer@tum.de}
%\and
%Team Member 5\\
%{\tt\small fifth@i1.org}
}


\maketitle
%\thispagestyle{empty}

\section{Introduction}
We call super-resolution (SR) the task of estimating a high-resolution
(HR) image from its low-resolution (LR) counterpart. Recent work with
optmizition-based methods largely focuses on minimizing the mean
squared reconstruction error. This results in high peak
signal--to--noise ratios, but they often have problems with modelling
high--frequency details. The results are often too smooth. In our work
we tried to tackle this problem using generative adversarial networks
(GANs).

% Dieser Abschnitt als kleine Einleitung, grober Überblick
% über: was ist unsere Problemstellung, was gibt es dazu bereits bzw wie
% wird versucht das Problem zu lösen und was ist die Schwierigkeit
% dabei.

\section{GANs}
GANs consist of two different networks, a Generator Network and a
Discriminator Network. The concept behind this is that the generative
network estimates a super-resolved image from its LR version with the
goal to become highly similar to real images that the discriminator
network fails to distiguish.  %Hier würde ich im Poster das Bild der
Netword Architektur einfügen

Therefore we optimize the discriminator network $D_{\Theta_D}$ in an
alternating manner along with the generative network $G_{\Theta_G}$ to
solve the adversarial min-max problem:\\
$G_{\Theta_G}$ to solve adversarial min-max problem:
\begin{align*}
  min_{\Theta_G} max_{\Theta_G} &\mathbb{E}_{I^{HR} \backsim p_{\text{train}}(I^{HR})} [\text{log} D_{\Theta_D}(I^{HR})] \\
  +&\mathbb{E}_{I^{HR} \backsim p_G(I^{LR})} [\text{log} (1-G_{\Theta_G}(I^{LR}))]
\end{align*}
% schaut bisschen komisch aus mit der Formel über die 2 Seiten
% (den Satz mit der Formel würde ich auf dem Poster rauslassen,
% außerdem auf die Netzwerk Architektur der beiden Netze nicht näher
% eingehen (sind halt deep nets mit conv layern))
The perceputal loss $l^{SR}$ we defined as weighted sum of a content loss and an discriminative loss component:\\
\begin{equation*}
  l^{SR}=\alpha l^{SR}_{MSE} + \beta l^{SR}_{VGG16_19/i.j} + \gamma l^{SR}_{D}
\end{equation*}
More precisely, the content loss components are defined as follows:
\begin{align*}
  l^{SR}_{MSE}       &= \frac{1}{r^2WH}\sum_{x=1}^{rW}\sum_{y=1}^{rH}(I^{HR}_{x,y}-G_{\theta_G}(I^{LR})_{x,y})^2 \\
  l^{SR}_{VGG1619/i.j}&=\frac{1}{W_{i,j}H_{i,j}} \sum_{x=1}^{W_{i,j}}\sum_{y=1}^{H_{i,j}}(\Phi_{i,j}(I^{HR})_{x,y}-\Phi_{i,j}(G_{\Theta_G}(I^{LR}))_{x,y}^2
\end{align*}
the discriminative loss as follows
\begin{equation*}  
  l^{SR}_{D}=\sum_{n=1}^N -\text{log}D_{\Theta_D}(G_{\Theta_G}(I^{LR}))
\end{equation*}
\section{Setup}
\label{sec:setup}

% what we used for our work

\subsection{Dataset}
\label{sec:data}

For training we used the PASCAL VOC Dataset\cite{pascal-voc-2012} with
more than 10,000 images as well as the NITRE
Dataset\cite{Agustsson_2017_CVPR_Workshops} with 800 images.
% our datasets

\subsection{Networks}
\label{sec:nets}

We used pretrained VGG networks in different configurations. The best
results were obtained when we used a VGG1619 network. We also
considered a VGG 16 network. Although faster at training it did not
yield results of equal quality.

% the Networks, e.g. VGG16, VGG19, VGG16/19

\section{Results}
\label{sec:results}

% our results => main section
% discriminatro may be omitted
% better nets work better
% ...

\section{Conclusion}
\label{sec:conclusion}

% something about what we learned

\appendix
%%% Appendix
% include images
% include graphs

{
  \nocite{*}                      % also those references without \cite{·}
  \small
  \bibliographystyle{ieee}
  \bibliography{bib}
  
}
\end{document}